% !TEX root = ../main.tex
% Included by MAIN.TEX

%--------------------------------------------------
% Fonts and page setup
%--------------------------------------------------

% Default font
\usepackage{palatino}
\usepackage{underscore}
% Enable special PostScript fonts (optional)
% \usepackage{pifont}

% Manipulate the footer
\usepackage{scrlayer-scrpage}
\usepackage{scrhack}
\pagestyle{scrheadings}
\ifoot[\footertext]{\footertext} % \footertext set in INFO.TEX

% Set the font for the section headings
\usepackage[utf8]{inputenc}
\usepackage[spanish,es-tabla]{babel}
\usepackage{tocloft}
\usepackage{blindtext}
\usepackage{pdfcomment}
\usepackage{enumitem}
\setlist{nosep}
\setlength\cftparskip{0.3\baselineskip}
\setlength\cftbeforechapskip{0pt}
\linespread{1.077}
\usepackage{makecell}
\usepackage[titletoc]{appendix}
\renewcommand{\sectfont}{\normalfont \bfseries}
\hyphenpenalty=750
\hypersetup{bookmarksdepth=4}
\usepackage{translator}
\usepackage{pgfgantt}

% Conditional commands in LaTeX documents, used for the \clearemptydoublepage.
\usepackage{ifthen}
%% \newboolean{twoside}
%% \setboolean{twoside}{false}

% Typeset text in multiple columns (optional)
% \usepackage{multicol}

% Rotation tools, including rotated full-page floats (optional)
\usepackage{rotating}


%--------------------------------------------------
% Document structure
%--------------------------------------------------

% Standard LaTeX package for creating indexes
\usepackage{makeidx}

% Pro­duce hy­per­text links in the doc­u­ment (recommended)
\usepackage{hyperref}

%--------------------------------------------------
% Bibliography
%--------------------------------------------------

% Set the bibliography style (default: plain)
\bibliographystyle{plain}

% Special biblography package (nice to have)
% \usepackage{natbib}


%--------------------------------------------------
% Graphics and floats
%--------------------------------------------------

% Enhanced support for graphics (recommended)
\usepackage{graphicx}
% Path to the figures directory (default: {figures/})
% Multiple entries are allowed, e.g. {{figures1/}{figures2/}}.
\graphicspath{{figures/}}

% Improved interface for floating objects (optional)
\usepackage{float}

% To use the subfigures (optional)
\usepackage{subcaption}


%--------------------------------------------------
% Mathematics
%--------------------------------------------------

% AMS mathematical facilities for LaTeX (recommended)
\usepackage{amsmath}

% TeX fonts from the American Mathematical Society (recommended)
\usepackage{amsfonts}

\usepackage{inconsolata}
% Some extra math symbols (optional)
% \usepackage{amssymb}

% Extended maths fonts for LaTeX (optional)
% \usepackage{yhmath}

% Provide math delimiters whose size can be computed automatically (optional)
% \usepackage{commath}


%--------------------------------------------------
% Source code and algorithms
%--------------------------------------------------

% Source code typesetting
% \usepackage{listings} % (optional - alternative)
%\usepackage[newfloat]{minted} % (recommended)
% Set global Minted options
%\setminted{linenos, autogobble, frame=lines, framesep=2mm, fontsize=\footnotesize}
%\usemintedstyle{autumn}
% Inline C++ (optional)
%\newcommand{\incpp}[1]{\mintinline{c++}{#1}}
% \newenvironment{code}{\captionsetup{type=listing}}{}
% \SetupFloatingEnvironment{listing}{name=Código}

% Typeset algorithms - pseudocode (optional)
% \usepackage[ruled,vlined]{algorithm2e}
% \usepackage{algorithmic}
% \usepackage{algorithmicx}
% \usepackage{algpseudocode}
% Normal arrow comments
% \algrenewcommand{\algorithmiccomment}[1]{\hfill$\rightarrow$ #1}


%--------------------------------------------------
% Tables
%--------------------------------------------------

% Tables (optional)
\usepackage{booktabs}
\usepackage{longtable}

% Add color to LaTeX tables (optional)
\usepackage{colortbl}

% Create tabular cells spanning multiple rows (optional)
\usepackage{multirow}

\usepackage{tikz}

%--------------------------------------------------
% Color
%--------------------------------------------------

% Use colors
\usepackage{xcolor}
\definecolor{pinegreen}{rgb}{0.0, 0.47, 0.44}
% You may find all the pre-defined colors in
% https://en.wikibooks.org/wiki/LaTeX/Colors#Predefined_colors

% Color for the hyperlinks (e.g. table of contents)
\def\colorLinks{pinegreen}
% Color for the web links
\def\colorUrl{pinegreen}
% Color for the citations
\def\colorCitations{pinegreen}

%--------------------------------------------------
% PDF output
%--------------------------------------------------

% Adjust the color of the links
\hypersetup{
  linkcolor=\colorLinks,%
  urlcolor=\colorUrl,%
  citecolor=\colorCitations
}

% Disable the coloring of the links when printing.
% Requires a compatible PDF reader.
\usepackage[ocgcolorlinks]{ocgx2}[2017/03/30]

% PDF Metadata
\hypersetup{
  pdftitle={\metaTitle},%
  pdfauthor={\metaAuthor},%
  pdfkeywords={\metaKeywords},%
  pdfsubject={\metaSubject}
}

% Create XMP Metadata (uses the values from hyperref)
\usepackage{hyperxmp}

% Make thumbnails (optional)
% \usepackage{thumbpdf}


%--------------------------------------------------
% Other settings
%--------------------------------------------------

% Define commands that appear not to eat spaces (optional)
\usepackage{xspace}
