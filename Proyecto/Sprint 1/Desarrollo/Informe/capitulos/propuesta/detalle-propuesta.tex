\section{Detalle de la propuesta}
\label{section:propuesta}

\subsection{Descripción de la situación objetivo}
\label{subection:prop:situacion-objetivo}

\Blindtext[1]

\subsection{Recursos para el desarrollo del proyecto}
\label{subsection:prop:recursos}
\subsubsection{Humanos, financieros, tiempo, información}
Para el desarrollo del proyecto se requieren los siguientes recursos:

\paragraph{Humanos}
\begin{description}[leftmargin=2cm,labelindent=1cm]
  \setlist[description,2]{style=nextline}
\item[Estudiantes] Rol ejecutor
  \begin{description}
    \item[\firstauthor] Profesión/ experiencia/ experticia.
    \item[\secondauthor] Profesión/ experiencia/ experticia.
    \item[\thirthauthor] Profesión/ experiencia/ experticia.
    \item[\fourthauthor] Profesión/ experiencia/ experticia.
  \end{description}
\item[Profesores] Rol asesor y evaluador
  \begin{description}
    \item[María del Pilar Villamil] Ph.D. en Systemes et Logiciels.\\
      Profesora Asociada Universidad de Los Andes.
    \item[Germán Enrique Bravo Córdoba] D.E.A. en Informatique.\\
      Profesor Asociado Universidad de Los Andes
  \end{description}
\item[Recursos de la Organización] Rol usuario / experto
  \begin{description}
    \item[Nombre 1] Rol en la empresa.\\Rol en el proyecto (Experto de negocio / Experto técnico / Cliente directo del producto / Asesor).
    \item[Nombre 2] Rol en la empresa.\\Rol en el proyecto (Experto de negocio / Experto técnico / Cliente directo del producto / Asesor).
    \item[Nombre 3] Rol en la empresa.\\Rol en el proyecto (Experto de negocio / Experto técnico / Cliente directo del producto / Asesor).
    \item[Nombre 4] Rol en la empresa.\\Rol en el proyecto (Experto de negocio / Experto técnico / Cliente directo del producto / Asesor).
  \end{description}
\end{description}

\paragraph{Financieros} Costos%El proyecto no contempla costos adicionales a las licencias con que ya cuenta el área de tecnología. No se incurrirán en costos adicionales desde ninguna de las partes involucradas. % poner acá algo distinto si es pertinente.

\paragraph{Tiempo} tiempo %La duración del proyecto desde su consolidación hasta la entrega de documentación final es de 15 semanas. Para el desarrollo, despliegue y validación se presupuestan 13 semanas.

\paragraph{Información}
Describir los datos acá

\subsubsection{Herramientas disponibles para el desarrollo del proyecto}
Se cuentan con herramientas de código libre (o en su defecto gratuitas) para el
desarrollo del desarrollo como son:
\begin{description}[labelindent=1cm]
  \item [python] Lenguaje de desarrollo
  \item [django] Framework de desarrollo web
  \item [pandas] Librería de manipulación de datos
  \item [docker] Herramienta para manejo de contenedores en ambientes
    virtualizados
  \item [scikit-learn] Librería de machine learning para python
  \item [d3.js] Librería para visualización de datos
  \item [git + github.com] Herramienta de hosting para proyectos
    versionados
  \item [Alguna base de datos] Base de datos relacional/NoSQL etc
\end{description}

Se cuentan también con herramientas corporativas, para las cuales el cliente
cuenta con licencia:
\begin{description}[labelindent=1cm]
  \item [Herramienta 1] Descripción
  \item [Herramienta 2] Descripción
  \item [Herramienta 3] Descripción
\end{description}

\subsubsection{Infraestructura necesaria para el desarrollo del proyecto}
El desarrollo del proyecto requiere:
\begin{itemize}
  \item Describir servidores de bases de datos
  \item Describir servidores para despliegue
  \item Describir servidores para desarrollo
  \item Describir hosting usado
\end{itemize}

\subsection{Requerimientos de información}
\label{subsection:prop:info}

\subsubsection{Descripción general de los datos}
Brevísima introducción.
\begin{enumerate}[label=RI\arabic*] % ids de la forma RI1, RI2
  \item \textbf{Datos 1}
    \label{req_info:nombre_requerimiento} Descripción de qué son y qué elementos deben tener. Incluir descripción de campos mínimos, restricciones de fecha, formato, etc.
  \item \textbf{Datos 1}
    \label{req_info:nombre_requerimiento2} Descripción de qué son y qué elementos deben tener

\end{enumerate}
Para satisfacer estos requerimientos se cuentan con las siguientes fuentes:
\begin{itemize}
  \setlist[description]{style=nextline, font=\monobold, leftmargin=2cm,
    labelindent=1cm}
  \item \monobold{nombre_fuente_1} Satisface el requerimiento \ref{req_info:nombre_requerimiento}.
    Descripción del tipo de la fuente, dueño y contenidos:
    \begin{description}
      \item[tabla_1] Descripción de la información.\\
        Volumen esperado: XY registros.
      \item[tabla_2] Descripción de la información.\\
        Volumen esperado: XY registros.
    \end{description}
  \item \monobold{nombre_fuente_2} Satisface el requerimiento
    \ref{req_info:nombre_requerimiento2}.
    Descripción del tipo de la fuente, dueño y contenidos:
    \begin{description}
      \item[tabla_1] Descripción de la información.\\
        Volumen esperado: XY registros.
      \item[tabla_2] Descripción de la información.\\
        Volumen esperado: XY registros.
    \end{description}
\end{itemize}

\subsubsection{Identificación del ciclo de vida de los datos}
\begin{enumerate}
  \item etapa 1
  \item etapa 2
  \item etapa 3
  \end{enumerate}

\subsubsection{Calidad de los datos }
Descripción de la calidad

\subsubsection{Necesidades de preprocesamiento de datos, esfuerzo estimado y
costos asociados}
Se estima un esfuerzo Xtiempo y Npersonas para completar el procesamiento de los datos. Descripción de cosas por hacer.
\subsubsection{Fecha de disponibilidad de los datos necesarios para el
desarrollo del proyecto}
Los datos estarán disponibles completamente en X fecha.

\subsection{Resultados esperados}
\label{subsection:prop:resultados}

\subsubsection{Entregables a la organización}
Describir los entregables

A continuación, en la tabla \rtab{detalle_entregables1} se explica en detalle en qué consisten los entregables:

\insertlongtable{Entregables a la organización}{C{3.5cm} C{2.6cm} p{0.55\linewidth}}{
\textbf{Entregable} & \textbf{Tipo}  &  \textbf{Descripción}}{

	Nombre entregable & Documento / reporte / desarrollo & Descripción. \\
	\midrule

	Nombre entregable & Documento / reporte / desarrollo & Descripción. \\
	\midrule

	Nombre entregable & Documento / reporte / desarrollo & Descripción. \\
	\midrule

	Nombre entregable & Documento / reporte / desarrollo & Descripción. \\
	\midrule

	Nombre entregable & Documento / reporte / desarrollo & Descripción. \\

}{detalle_entregables1}



\subsubsection{Entregables a la Universidad}
Los siguientes son los tipos de entregables que se van a entregar a la Universidad como requisito para la aprobación del curso Proyecto Final:

\begin{itemize}
	\item Aprobación de la empresa o usuario final.
	\item Documento final del proyecto de grado.
	\item Informes de avance del proyecto de grado.
\end{itemize}

A continuación, en la tabla \rtab{detalle_entregables2} se explica en detalle en qué consisten los entregables:

\insertlongtable{Entregables a la universidad}{C{3.5cm} C{2.6cm} p{0.55\linewidth}}{
\thead{Entregable} & \thead{Tipo}  &  \thead{Descripción}}{

	Aval de la empresa & Aprobación de la empresa & Documento firmado por la empresa usuaria de la solución mostrando su aval hacia el proyecto. \\
	\midrule

	Propuesta del proyecto & Documento de grado & Documento que contiene el contexto del proyecto, la definición del problema, los objetivos, el alcance, los riesgos, participantes, plan de desarrollo, entre otros. \\
	\midrule

	Acta de constitución del proyecto & Documento de grado / Aprobación de la empresa & Documento inicial que establece el problema a resolver,  los objetivos, entre otros. Debe estar firmado por los participantes del lado del equipo de desarrollo de la solución y del lado de la empresa. \\
	\midrule

	Informe de avance 1 & Informe de avance & Diseño inicial de la solución, selección tecnológica, requerimientos críticos resueltos.Informe de calidad de datos, retos de análisis y transformación de información, plan de pruebas, estrategia de validación. Cronograma detallado de desarrollo, con responsables. \\
	\midrule

	Informe de avance 2 & Informe de avance & Estado de avance. Diseño detallado, avance de implementación, prototipo inicial, prueba de concepto. Procesamiento de información. Demostración y presentación. \\
	\midrule

	Informe de avance 3 & Informe de avance & Estado de avance. Diseño final del producto y de la solución. Implementación completa, pruebas sintonización, resultados de validación, completitud de la solución al problema. \\
	\midrule

	Evaluaciones de avance & Aprobación de la empresa & Aprobación del avance por parte del experto de negocio y del responsable del proyecto en la organización. \\
	\midrule

	Entrega Final & Documento de grado & Presentación final del proyecto. Informe final. Sustentación y demostración. \\
	\midrule

	Acta de entrega a la empresa & Aprobación de la empresa & Acta de entrega del proyecto a la empresa firmada por el responsable del proyecto en la organización y por el experto de negocio en la misma. \\

}{detalle_entregables2}

\subsection{Impacto y beneficios}
\label{subsection:prop:impacto}

\tinylipsum
