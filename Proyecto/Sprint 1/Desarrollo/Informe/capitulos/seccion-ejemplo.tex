\addchap{Sección de referencia}

\section*{Citaciones a bibliografía}
\tinylipsum Tomado de \cite{latex}.

\section*{Notas al pie}
\tinylipsum \footnote{Properly formatted clickable URL: \url{https://www.tum.de/}}

\section*{Listas}
\tinylipsum

\tinylipsum
\begin{enumerate}
  \item \tinylipsum
    \begin{enumerate}
      \item \tinylipsum
      \item \tinylipsum!
    \end{enumerate}
  \item \tinylipsum
    \begin{itemize}
      \item list
      \item things!
    \end{itemize}
  \item \tinylipsum!
\end{enumerate}

\begin{itemize}
  \item list
  \item all
  \item the
  \item things!
\end{itemize}
\tinylipsum

\section*{Comentarios al margen}
Texto acá \comment{Comentarios allá} \tinylipsum

\section*{Figuras}
% Filename, Caption, Label, Width % of textwidth
\tinylipsum
\insertfigure{imagenes/edt/edt-nivel1.pdf}%
{Plot de una función}{fig:fnplot}{0.7}
\tinylipsum

\section*{Tablas}
% si es necesario que la tabla se divida entre dos páginas cambiar inserttable por insertlongtable
\inserttable{Tabla de ejemplo}{C{11.7cm} C{2.6cm}}{
\thead{N1} & \thead{N2}}{

	R1 & R2 \\
    \midrule
	R3 & R4 \\
}{label-ref-tabla}

\tinylipsum

\section*{Anotaciones sobre el pdf}

\tinylipsum \feedbackans{Se agregó un lipsum no tan grande, pero se agregó} \tinylipsum \tinylipsum

\section*{Referencias a partes del texto}
Acá referenciamos la \rsec{section:lecciones-aprendidas} dentro del \rchap{chapter:conclusions}
